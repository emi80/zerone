\documentclass[12pt]{article}
\usepackage[toc,page]{appendix}
\usepackage{amsmath}
\usepackage{amsfonts}
\usepackage{graphicx}
\usepackage{url}
\usepackage{cite}
\title{Supplementary information \\
\large{Zerone: a ChIP-seq discretizer for multiple replicates \\
with built-in quality control}}
\author{Guillaume Filion and Pol Cusc\'o}

\begin{document}
\maketitle

%In the text, we often refer to the gamma and polygamma functions.
%Euler's $\Gamma$ function is defined as
%$\Gamma(\alpha) = \int_0^{\infty}t^{\alpha+1}e^{-t}dt$.
%The digamma function, noted $\psi(\alpha)$ is the
%derivative of $\log \Gamma(\alpha)$, and the trigramma function,
%noted $\psi'(\alpha)$ is the derivative of the digamma function.

We start with some generalities about Hidden Markov Models and then
derive a model targeted to ChIP experiments with replicates.
  
\section{Hidden Markov Models}

    We will consider only discrete Hidden Markov models (HMMs) and
    will simply refer to them as Hidden Markov model, without mention
    of the term `discrete' for simplicity.  HMMs are defined by 

    \begin{enumerate}
      \item a set $S$ of $m$ states numbered from 1 to $m$,
      \item an initial state probability distribution $\nu$, which
      gives the probabilities that the system is initially in state $i$,
      \item an $m \times m$ transition matrix $Q$ which contains the
      probabilities $Q(i,j)$ that the system goes from state $i$ to
      state $j$,
      \item $m$ distributions denoted $g_i$ $(i = 1, \ldots, m)$, which
      give the emission probabilities in the different states.
    \end{enumerate}

\subsection{The HMM formalism applied to ChIP-seq}
\label{sec:HMM_formalism}

    The output of ChIP-seq experiments consists of genomic profiles
    that can be thought of as ordered windows of equal size. Each
    window is associated to a certain number of read counts coming
    from different replicate experiments or negative controls.

    HMMs are particularly adapted to this framework. The read counts
    associated to each window can be thought of as the observable
    emissions, whereas the unobservable states can be thought of
    the possible processes ongoing on those windows. Typically, the
    states may correspond to the events ``the protein of interest
    is bound on the window'' and ``the protein of interest is not
    bound on the windown''. There may be more states, and they do
    not have to correspond to a protein binding event (most notably,
    they can correspond to the present of some histone modifications).

    The rest of this section pertains to general HMMs and will not
    make any hypothesis regarding the nature of the states and the
    emissions, however it can be useful for the intuition to think
    about states are chromatin states, and emissions as read counts.

\subsection{The Forward-Backward algorithm}

    For a sequence of emissions $y_0, \ldots, y_n$, the likelihood
    of the state sequence $i_0, \ldots, i_n$ is proportional to

    $$ \nu(i_0)g_{i_0}(y_0)
       \prod_{k=1}^n Q(i_{k-1},i_k)g_{i_k}(y_k). $$

    By summing over all possible combinations of states, we obtain
    the normalizing constant $L_n$ such that

    \begin{equation}
       L_n = \sum_{i_0 \in S, \ldots, i_n \in S} \nu(i_0)g_{i_0}(y_0)
       \prod_{k=1}^n Q(i_{k-1},i_k)g_{i_k}(y_k).
    \end{equation}

    We denote $\phi_{k|n}(i)$ the probability that the system is in
    state $i$ at time $k$ given the emissions $y_0, \ldots, y_n$. If
    we call $S_n(k,i)$ the set of $n$-tuples $(i_0, \ldots, i_n)$
    such that $i_k = i$, the value of $\phi_{k|n}(i)$ comes as

    $$ \phi_{k|n}(i) = \frac{1}{L_n}
       \sum_{(i_0, \ldots, i_n) \in S_n(k,i)}
       \nu(i_0)g_{i_0}(y_0) \prod_{l=1}^n Q(i_{l-1}, i_l)
       g_{i_l}(y_l). $$

    We now introduce $\alpha_k(i)$ the probability that the
    system is in state $i$ at time $k$ given the emissions
    $y_0, \ldots, y_k$, and the $\beta_{k|n}(\cdot)$ the numerical
    function such that $\phi_{k|n}(i) = \alpha_k(i)\beta_{k|n}(i)$.
    
    \begin{align*}
      \alpha_k(i) &= \frac{1}{L_k}
      \sum_{i_0=1}^m \cdots \sum_{i_{k-1}=1}^m
      \nu(i_0)g_{i_0}(y_0) \prod_{l=1}^{k-1} Q(i_{l-1},i_l) g_{i_l}(y_l)
      Q(i_{k-1}, i)g_i(y_k) \\
      \beta_{k|n}(i) &= \frac{L_k}{L_n}
      \sum_{i_{k+1}=1}^m \cdots \sum_{i_n=1}^m
      Q(i, i_{k+1})g_{i_{k+1}}(y_{k+1})
      \prod_{l=k+2}^n Q(i_{l-1}, i_l)g_{i_l}(y_l)
    \end{align*}

    To preserve the equality $\phi_{k|n}(i) = \alpha_k(i)\beta_{k|n}(i)$
    for every $k$, we set by definition $\beta_{n|n}(i) = 1$.
    From the equations above, we draw the following recursive
    equations:

    \begin{align} \alpha_k(i) &= \frac{L_{k-1}}{L_k}
      \sum_{j=1}^m \alpha_{k-1}(j) Q(j,i) g_i(y_k) \label{alpha} \\
      \beta_{k|n}(i) &= \frac{L_k}{L_{k+1}}
      \sum_{j=1}^m Q(i,j) g_j(y_{k+1})
      \beta_{k+1|n}(j). \label{beta}
    \end{align}

    Equations (\ref{alpha}) and (\ref{beta}) are the basis of the
    Forward-Backward algorithm to compute $\phi_{k|n}(i)$. The terms
    $\alpha_k(i)$ can be recursively computed from $k=0$ to $k=n$
    with equation (\ref{alpha}), and the terms $\beta_{k|n}(i)$
    can be computed from $k=n-1$ to $k=0$ with equation (\ref{beta}).
    The terms $\phi_{k|n}(i)$ are then found as the product
    $\alpha_k(i)\beta_{k|n}(i)$.

    We now turn to the term $\phi_{k-1,k|n}(i,j)$, which is by
    definition the probability that the system is in state $i$ at
    time $k-1$ and in state $j$ at time $k$ given $y_0, \ldots, y_n$.
    If we call $S_n(k,i,j)$
    the set of $n$-tuples $(i_0, \ldots, i_n)$ such that
    $i_{k-1} = i$ and $i_k = j$, we get

    \begin{align}
      \phi_{k-1,k|n}(i,j) &= \frac{1}{L_n}
       \sum_{(i_0, \ldots, i_n) \in S_n(k,i,j)}
       \nu(i_0)g_{i_0}(y_0) \prod_{l=1}^n Q(i_{l-1}, i_l)
       g_{i_l}(y_l) \nonumber \\
        &= \frac{L_{k-1}}{L_k}
       \alpha_{k-1}(i) Q(i,j) g_j(y_k) \beta_k(j). \label{phiQ}
    \end{align}

    When the $\alpha_k(i)$ and the $\beta_{k|n}(i)$ have been
    computed by the Forward-Backward algorithm, we also have access
    to the $\phi_{k-1,k|n}(i,j)$ by using formula (\ref{phiQ}).
    
\subsection{The Baum-Welch algorithm}
\label{sec:Baum-Welch}

    The Baum-Welch algorithm is the special case of the EM algorithm
    applied to HMMs. Let us consider the general case of the triplet
    $(X, Z, \theta)$ where the variable $X$ is observed, $Z$ is not
    observed, and $\theta$ is the set of parameters of the distribution
    of $(X,Z)$. The full likelihood $\mathcal{L}_0(X, Z, \theta)$
    cannot be computed because the value of $Z$ is unknown.

    To find the value of $\theta$ that maximizes the full likelihood,
    we introduce an iterative procedure where the values of the
    parameter are updated upon each iteration. The current value of
    $\theta$ is noted $\theta^{(t)}$, and we compute the expected
    complete log-likelihood $\mathcal{Q}(\theta|\theta^{(t)})$
    assuming the current value of $\theta$ (note the difference
    between the intermediate quantity of the EM $\mathcal{Q}$ and
    the transition matrix $Q$).

    $$ \mathcal{Q}(\theta|\theta^{(t)}) =
      E_{Z|X, \theta^{(t)}} \left\{
      \log \mathcal{L}_0(X, Z, \theta^{(t)}) \right\}$$

    This computation is called the E-step. The notations mean that
    the expectation is taken over the variable $Z$, assuming that
    it is conditional on the observed values of $X$ and that the
    parameters of the distribution are given by $\theta^{(t)}$.
    The E-step is followed by the
    M-step, in which $\theta^{(t+1)}$ is set to the value of
    $\theta$ that maximizes $\mathcal{Q}(\theta|\theta^{(t)})$.

    In the case of HMMs, the variable that is not observed is the
    sequence of states. The set of parameters $\theta^{(t)}$
    represents the transition probabilities (the matrix $Q$)
    and the parameters of the $m$ distributions of the emissions.

    The log-likelihood of the state sequence $(i_0, \ldots, i_n)$
    is

    $$ \log \nu(i_0) + \sum_{k=1}^n \log Q(i_{k-1}, i_k)
      + \sum_{k=0}^n \log g_{i_k}(i_k, \theta). $$

    The addition of $\theta$ to the terms above emphasizes that they
    depend on the value of the parameters. To compute
    $\mathcal{Q}(\theta|\theta^{(t)})$, we need to take the
    expectation of the above over the state sequence
    conditionally on $y_0, \ldots, y_n$ and assuming that the parameters
    are given by $\theta^{(t)}$.

    \begin{align}
      \mathcal{Q}(\theta|\theta^{(t)}) &=
      E_{\theta^{(t)}} \left\{ \log \nu(i_0)
      \big| y_0, \ldots, y_n \right\} + \nonumber \\
      &\sum_{k=1}^n E_{\theta^{(t)}} \left\{
        \log Q(i_{k-1}, i_k)\big| y_0,
            \ldots, y_n \right\} + \label{eq:QEM} \\
      &\sum_{k=0}^n E_{\theta^{(t)}} \left\{
        \log g_{i_k}(y_k, \theta) \big| y_0, \ldots, y_n \right\}
        \nonumber
    \end{align}

    In practice, the first term of (\ref{eq:QEM}) will often not depend
    on $\theta$ so it will not contribute to the evaluation.
    The third term can be rewritten as

    $$\sum_{k=0}^n\sum_{i=1}^m \phi_k(i) \log g_{i_k}(y_k, \theta).$$

    This term depends on the emission probabilities, and nothing
    can be said about it in general terms because they differ
    between different models. But the second term depends only on
    the transition probabilities, which are present in every HMM,
    and it can be solved in general. First we notice that

    \begin{align*}
      &E_{\theta^{(t)}} \left\{ \log Q(i_{k-1}, i_k)
      \big| y_0, \ldots, y_n \right\} = \\
      &E_{\theta^{(t)}} \left\{ \sum_{i=1}^m\sum_{j=1}^m 
      1_{\{(i_{k-1}, i_k) = (i,j)\}} \log Q(i,j)
      \big| y_0, \ldots, y_n \right\} = \\
      &\sum_{i=1}^m\sum_{j=1}^m E_{\theta^{(t)}} \left\{
      1_{\{(i_{k-1}, i_k) = (i,j)\}} \big| y_0, \ldots, y_n \right\}
      \log Q(i,j)
    \end{align*}

    Remember that by definition $E_{\theta^{(t)}} \left\{
    1_{\{(i_{k-1}, i_k) = (i,j)\}} \big| y_0, \ldots, y_n \right\}$
    is $\phi_{k-1,k}(i,j)$, so that we can rewrite the second term
    of (\ref{eq:QEM}) as

    $$ \sum_{k=1}^n\sum_{i=1}^m\sum_{j=1}^m \phi_{k-1,k}(i,j)
      \log Q(i,j). $$

    The values of $\phi_{k-1,k}(i,j)$ are computed during the
    E-step by the Forward-Backward algorithm.
    The terms $Q(i,j)$ are part of $\theta$ and are thus updated
    during the M-step. By using Lagrange multipliers, we can show
    that the update values are

    $$ Q(i,j)^{(t+1)} = \frac{\sum_{k=1}^n \phi_{k-1,k}(i,j)}
      {\sum_{k=1}^n\sum_{l=1}^m\phi_{k-1,k}(i,l)}. $$
    
    To complete the Baum-Welch algorithm, we need to compute the last
    term of (\ref{eq:QEM}), which requires making a model for the
    emissions.

\section{Zero-Inflated Negative Multinomial}
\label{sec:nb}

The Baum-Welch algorithm provides a general framework to estimate
the transition probabilities and the emission parameters. However,
the detail of the estimation depends on the emission model. Here
we give some general results about the negative multinomial and
the zero-inflated negative multinomial distributions that will
be useful to setup a model for emissions in ChIP-seq experiments.

\subsection{The Gamma-Poisson process}
\label{sec:gamma_poisson}

    In what follows, $y$ is a non negative integer (an element of
    $\mathbb{N}$). Let $Y$ be a discrete random variable distributed
    as a Poisson distribution with parameter $\lambda$. The
    probability that $Y$ is equal to $y$ is by definition

    \begin{equation}
\label{eq:poisson}
      P(Y=y) = e^{-\lambda} \frac{\lambda^y}{y!}.
    \end{equation}

    Let us now assume that $\lambda$ is itself a random variable,
    such that the above equality is actually $P(Y=y | \lambda)$.
    If $\lambda$ is independent of $Y$ and has a Gamma distribution
    with parameters $\alpha$ and $\beta$, the joint distribution of
    $Y$ and $\lambda$ is the product of the two distributions, that
    is

    $$ e^{-\lambda} \frac{\lambda^y}{y!}
         \frac{1}{\Gamma(\alpha)\beta^{\alpha}} e^{-\lambda/\beta}
         \lambda^{\alpha-1}. $$

    The marginal distribution of $Y$, \textit{i.e.} $P(Y=y)$, is found
    by integrating the equality above over $\lambda$.

    \begin{align}
\label{eq:nb_distrib}
      P(Y=y) &= \frac{1}{\Gamma(\alpha)\beta^{\alpha}y!}
         \int_0^{+\infty} e^{-\lambda(1+1/\beta)} \lambda^{\alpha+y-1}
         d\lambda \nonumber \\
        &= \frac{\Gamma(\alpha+y)}{\Gamma(\alpha)\beta^{\alpha}
          (1+1/\beta)^{\alpha+y} y!} \nonumber \\
        &= \frac{\Gamma(\alpha+y)}{\Gamma(\alpha)y!}
          \left(\frac{1}{1+\beta}\right)^{\alpha}
          \left(\frac{\beta}{1+\beta}\right)^y.
    \end{align}

    Equation (\ref{eq:nb_distrib}) is the expression of the negative
    binomial distribution, with one of the many possible
    parametrizations. We will refer to this distribution as a
    negative binomial with parameters $(\alpha, 1/(1+\beta))$.

    The Gamma-Poisson process can describe many phenomena
    because of the flexibility of the Gamma distribution. The
    $\alpha$ parameter of the Gamma distribution is often referred
    to as the ``shape'' parameter. For negative values of $alpha$,
    the distribution has a singularity at 0, whereas for positive
    values, the distribution has a single ``bump''. The $\beta$
    parameter is often referred to as the ``scale'' parameter
    because it represents a stretching of the curve along the x-axis
    that can fit different variances. As a result, the Gamma
    distribution is a good choice for almost every continuous
    distribution with positive values and a single mode. Combined
    to the Poisson distribution, it allows to fit almost every
    discrete distribution with positive values.

\subsection{The negative multinomial distribution}

    We now introduce the case of $r$ Poisson variables that are
    conditionally independent given $\lambda$. Intuitively, this
    means that $\lambda$ is drawn at random first, which sets the
    parameter of the $r$ Poisson distributions; the Poisson
    variables are then drawn from their respective distribution
    independently of each other. In other words, the conditional
    distribution can be written as follows

    \begin{equation}
\label{eq:conditional_nm}
      P(Y_1=y_1, \ldots, Y_r=y_r|\lambda) = 
      e^{-\gamma_1\lambda}\frac{(\gamma_1\lambda)^{y_1}}{y_1!}
      \ldots
      e^{-\gamma_r\lambda}\frac{(\gamma_r\lambda)^{y_r}}{y_r!}.
    \end{equation}

    Multiplying by the density of $\lambda$ and integrating as
    above, the marginal distribution of the vector
    $(Y_1, \ldots, Y_r)$ comes as

    \begin{align}
\label{eq:nm_distrib}
      P(Y_1=y_1, \ldots, Y_r=y_r) &=
      \frac{\Gamma(\alpha+y_1+\ldots+y_r)}
      {\Gamma(\alpha)y_1!\ldots y_r!}p_0^{\alpha}p_1^{y_1}
      \ldots p_r^{y_r}, \; \text{where}                \\
      p_0 &= \frac{1/\beta}{1/\beta+\gamma_1+\ldots+\gamma_r},
      \; \text{and} \nonumber \\
      p_i &= \frac{\gamma_i}{1/\beta+\gamma_1+\ldots+\gamma_r},
      \; \text{for} \; i = 1, \ldots, r. \nonumber
    \end{align}

    This distribution is called the negative multinomial, which
    the negative binomial is a special case of (for $r=1$). We will
    refer to it is as a negative multinomial with parameters
    $(\alpha, p_1, \ldots, p_r)$. It can be interpreted as the
    observations of a Gamma-Poisson process, where a common $\lambda$
    is drawn from a Gamma distribution, and $r$ variables are drawn
    from independent Poisson distributions with parameters
    $\gamma_i \lambda$ $(1 \leq i \leq r)$. The variables
    $Y_1, \ldots, Y_r$ are independent contionally on $\lambda$,
    but in section \ref{sec:marginal_nm} we prove
    that they are never unconditionally independent.

    The negative multinomial distribution has an alternative
    interpretation that sometimes proves useful. Suppose an urn
    contains black balls and balls of $r$ different
    colors in respective proportions $p_0, p_1, \ldots, p_r$, such
    that $p_0 + p_1 + \ldots + p_r =1$. If we draw balls with
    replacement from the urn until a black ball is drawn for the
    $k$-th time, the probability that the counts for the balls of
    each color are $(y_1, \ldots, y_r)$ is

    \begin{equation*}
    {k-1+y_1+\ldots+y_r \choose (k-1), y_1, \ldots, y_r}
      p_0^k p_1^{y_1} \ldots p_r^{y_r} =
    \frac{\Gamma(k+y_1+\ldots+y_r)}{\Gamma(k)y_1! \ldots y_r!}
      p_0^k p_1^{y_1} \ldots p_r^{y_r}.
    \end{equation*}

    This is formula (\ref{eq:nm_distrib}), where $\alpha$ has
    been replaced by $k$. The negative multinomial distribution is
    thus a generalization of the drawing process described above.
    From the ball and urn interpretation, we get that the observed
    counts $(y_1, \ldots, y_r)$ are twice smaller for a twice larger
    value of $p_0$ or for a twice smaller value of $\alpha$.

\subsection{Marginal distributions}
\label{sec:marginal_nm}

    To compute the marginal distributions of
    $(Y_1, \ldots, Y_r)$, we could sum over (\ref{eq:nm_distrib}),
    but it is simpler to sum over (\ref{eq:conditional_nm}) and then
    multiply by the density of $\lambda$ and integrate. The sum of
    (\ref{eq:conditional_nm}) over all indices distinct from
    $s \leq r$ is a Poisson term of the form of (\ref{eq:poisson})
    therefore, integrating over $\lambda$ yields an expression
    similar to (\ref{eq:nb_distrib}), namely

    \begin{align*}
    P(Y_s = y) &= \frac{\Gamma(\alpha+y)}{\Gamma(\alpha)y!}
      p_0^{*\alpha} p_s^{*y}, \; \text{where} \\
    p_0^* &= \frac{1/\beta}{1/\beta + \gamma_s}, \; \text{and} \\
    p_s^* &= \frac{\gamma_i}{1/\beta + \gamma_s}.
    \end{align*}
    
    Not surprisingly, we obtain a negative binomial distribution.
    More interestingly though, the parameters of this
    distribution are linked to the previous parameters
    by the equality $p_s^* / p_0^* = p_s / p_0$. This equality
    comes in handy to recompute the parameters of the negative
    multinomial distribution when variables are added or removed.

    In the light of the analogy with balls in an urn, this result
    is not surprising. The marginal distribution corresponds to
    the same process where some colors are removed. The proportion
    of the remaining colors change, but their relative ratios do
    not.

    To illustrate the use of this equality, we show with $r=2$ that
    the marginal variables of a negative multinomial distribution
    are never independent. This also holds for $r > 2$, which can be
    proved by induction from the observation that mutual independence
    entails pairwise independence.
    
    Assume that $(Y_1,Y_2)$ has a negative multinomial distribution
    and that it is not degenerate (all the parameters are strictly
    positive).  Let us fix $y_2 = 0$. The terms
    $P(y_1 = k, y_2 = 0)$ are proportional to
    $\Gamma(\alpha+k) p_1^k/k!$ and the terms $P(y_1 = k)P(y_2 = 0)$
    are proportional to $\Gamma(\alpha+k) p_1^{*k}/k!$ where
    $p_1^* = p_1 / (p_1 + p_0) < p_1$ so equality cannot hold
    for every $k \geq 0$. In conclusion, the joint distribution
    cannot be equal to the product of the marginal distributions.

    Note that the proof above assumes $p_0 > 0$, which is
    a consequence of $\beta < \infty$. So as long as $\lambda$ is 
    distributed according to a proper Gamma distribution, which
    is a defining feature of the negative multinomial distribution, the
    variables cannot be independent.

    From the marginal distributions we can compute the conditional
    distribution of $(Y_1, \ldots, Y_s)$ given $(Y_{s+1}, \ldots, Y_r)$
    (and similary the distribution of any set of variables given
    the complentary set). Using the same approach as above, the
    marginal distribution is a negative multinomial that can be
    found to be

    \begin{align*}
    P(Y_{s+1}=y_{s+1}, \ldots, Y_r=y_r) &= \\
      \frac{\Gamma(\alpha + y_{s+1} + \ldots + y_r)}
      {\Gamma(\alpha)y_{s+1}! \ldots y_r!} &p_0^{\alpha}p_{s+1}^{y_{s+1}}
      \ldots p_r^{y_r} \left(\frac{1}{p_0 + p_{s+1} + \ldots + p_r}
      \right)^{\alpha+y_{s+1} + \ldots + y_r}.
    \end{align*}

    The conditional distribution is computed as the ratio of the
    full distribution and the marginal distribution.

    \begin{align*}
      P(Y_1=y_1, \ldots, Y_s=y_s|Y_{s+1}=y_{s+1}, \ldots, Y_r=y_r) &= \\
      \frac{\Gamma(\alpha + y_1 + \ldots + y_r)}
      {\Gamma(\alpha+y_{s+1}+\ldots+y_r)y_1! \ldots y_s!}
      &q_0^{\alpha+y_{s+1}+\ldots+y_r}p_1^{y_1} \ldots p_i^{y_s},
    \end{align*}

    \noindent
    where we define
    $q_0 = p_0+p_{s+1}+\ldots+p_r = 1-(p_1+\ldots+p_s)$.
    In other words, the distribution of $(Y_1, \ldots, Y_s)$ given
    $(Y_{s+1}, \ldots, Y_r)$ is negative multinomial with parameters
    $(\alpha+y_{s+1}+\ldots+y_r, q_0, p_1, \ldots, p_s)$.

\subsection{Inference about $\alpha$ and $p_0$}
\label{sec:inference_alpha}

    We now turn our attention to the distribution of the sum of
    observations drawn from negative multinomial distribution. More
    precisely, if $(y_1, \ldots, y_r)$ is a random observation
    from a negative multinomial distribution with parameters
    $(\alpha, p_0, p_1, \ldots, p_r)$, we are interested in the
    distribution of $y_1 + \ldots + y_r$.

    Conditionally on a given value of $\lambda$, $Y_1, \ldots, Y_r$
    are independent and Poisson distributed. In regard of equation
    (\ref{eq:conditional_nm}), this means that the conditional value
    of the sum is Poisson with parameter
    $\lambda(\gamma_1 + \ldots + \gamma_r)$. The distribution of the
    sum is thus negative binomial with parameters $(\alpha, p_0)$,
    where the value of $p_0$ is as defined in (\ref{eq:nm_distrib}).

    This observation will be basis of the demonstration that the
    vector of marginal sums is a sufficient statistics for $\alpha$
    and $p_0$, which means that all the inference about these two
    parameters can be performed with the marginal sums. Let us
    consider a random sample of size $n$ drawn from a negative
    multinomial distribution with parameters
    $(\alpha, p_0, p_1, \ldots, p_r)$ and compute its distribution
    conditionally on the marginal sums. The observations consist of
    $n$ vectors of dimension $r$, denoted $(y_{k,1}, \ldots, y_{k,r})$,
    where $1 \leq k \leq n$, and we denote the associated marginal
    sum $y_k$. The likelihood of the sample is

    \begin{equation*}
      \prod_{k=1}^n \frac{\Gamma(\alpha+y_{k,1}+\ldots+y_{k,r})}
      {\Gamma(\alpha)y_{k,1}!\ldots y_{k,r}!}p_0^{\alpha}
      p_1^{y_{k,1}} \ldots p_r^{y_{k,r}}.
    \end{equation*}

    The likelihood of the marginal sums is
    \begin{equation*}
      \prod_{k=1}^n \frac{\Gamma(\alpha+y_{k,1}+\ldots+y_{k,r})}
      {\Gamma(\alpha)(y_{k,1}+\ldots+y_{k,r})!}p_0^{\alpha}
      (p_1+\ldots+p_r)^{y_{k,1}+\ldots+y_{k,r}}.
    \end{equation*}

    The conditional distribution of the observations is found by
    dividing these two values, which yields
    \begin{equation*}
      \prod_{k=1}^n \frac{(y_{k,1}+\ldots+y_{k,r})!}
      {y_{k,1}!\ldots y_{k,r}!}
      p_1^{*y_{k,1}} \ldots p_r^{*y_{k,r}},
    \end{equation*}

    \noindent
    where $p_s^* = p_s/(p_1 + \ldots + p_r)$. This expression
    does not depend on either $\alpha$ nor $p_0$, which proves
    that the $n$ marginal sums represent a sufficient statistic
    for $\alpha$ and $p_0$.

\subsection{Negative multinomial parameter estimation}
\label{sec:param_est_nm}

    The multinomial negative distribution can be fitted with the
    maximum likelihood method. Suppose that an IID random sample of
    size $n$ is available and denote the individual observations
    $(y_{k,1}, \ldots, y_{k,r})$, $1 \leq k \leq n$. The likelihood
    of the observations is

    \begin{equation*}
      L = \prod_{k=1}^n \frac{\Gamma(\alpha+y_{k,1}+\ldots+y_{k,r})}
        {\Gamma(\alpha)y_{k,1}!\ldots y_{k,r}!} p_0^{\alpha}
        p_1^{y_{k,1}}\ldots p_r^{y_{k,r}} .
    \end{equation*}

    According to section \ref{sec:inference_alpha}, we can estimate
    $\alpha$ and $p_0$ from the marginal sums of the observed sample,
    which we denote $y_1, \ldots, y_n$. The likelihood of the marginal
    sums is

    \begin{equation*}
      L = \prod_{k=1}^n \frac{\Gamma(\alpha+y_k}
        {\Gamma(\alpha)yk!} p_0^{\alpha}
        (1-p_0)^{y_k} .
    \end{equation*}

    In practice it is easier to maximize the log-likelihood
    $\ell = \log(L)$, which is as follows

    \begin{equation}
\label{eq:ml_nm}
      \ell = \sum_{k=1}^n \log\Gamma(\alpha+y_k) +
      \log\Gamma(\alpha) -\log(y_k!) + 
      \alpha \log(p_0) + y_k\log(1-p_0).
    \end{equation}

    The maximum is found by differentiating (\ref{eq:ml_nm}).

    \begin{equation*}
      \frac{\partial\ell}{\partial p_0} = \sum_{k=1}^n
      \frac{\alpha}{p_0}-\frac{y_k}{1-p_0} = 0.
    \end{equation*}

    The equation above can be rearranged to express $p_0$ as
    a function of $\alpha$

    \begin{equation}
\label{eq:sub_po_alpha}
      p_0 = \frac{\alpha}{\alpha + \bar{y}},
    \end{equation}

    \noindent
    where $\bar{y}$ is the mean of the marginal sums (\textit{i.e.}
    $\bar{y} = \sum_{k=1}^ny_k / n$). Differentiating with respect
    to $\alpha$, we now obtain

    \begin{equation*}
      \frac{\partial\ell}{\partial\alpha} = \sum_{k=1}^n
      \psi(\alpha+y_k) - \psi(\alpha) + \log(p_0).
    \end{equation*}

    We substitute (\ref{eq:sub_po_alpha}) in the equation above
    and obtain an expression that depends on $\alpha$ only.

    \begin{equation}
\label{eq:NR_f}
      f(\alpha) = n \Big(\log(\alpha) - \psi(\alpha) 
      - \log (\alpha + \bar{y}) \Big)
      + \sum_{k=1}^n \psi(\alpha+y_k) = 0.
    \end{equation}

    The equation above is solved by the Newton-Raphson method. For
    this, we use the update formula
    $\alpha^{(t+1)} = \alpha^{(t)} - f(\alpha^{(t)})/f'(\alpha^{(t)})$,
    which requires to differentiate $f$ and to choose an initial
    value $\alpha^{(0)}$.

    \begin{equation*}
      f'(\alpha) = n \left(\frac{\bar{y}} {\alpha(\alpha+\bar{y})}
      -\psi'(\alpha) \right)
      + \sum_{k=1}^n \psi'(\alpha+y_k).
    \end{equation*}

    The Newton-Raphson method is fast and converges after a few
    cycles. The initial value is usually chosen as
    $\alpha^{(0)} = 1$.
    Once $\alpha$ is known, the value of $p_0$ is set using
    equation (\ref{eq:sub_po_alpha}).

    To find the values of $p_1, \ldots, p_r$, we differentiate
    $\log(L)$ with respect to $p_s$ $(1 \leq s \leq r)$ and use
    Lagrange multipliers. It then appears that

    \begin{equation}
      p_s = \frac{p_0 \bar{y}_s}{\alpha},
    \end{equation}

    \noindent
    where $\bar{y}_s$ is the mean of the $s$-th variable,
    $\bar{y}_s = \sum_{k=1}^n y_{k,s}/n$.

\subsection{Zero-Inflation}

    The so called zero-inflated negative binomial (ZINB) is a mixture
    model with a negative binomial and a constant equal to 0. This 
    inflates the term $P(Y=0)$, whence the name of the distribution.
    Zero-inflated distributions are good models for overdispered
    data (which can occur as a consequence of distribution mixture).
    By definition, the numbers are drawn from a negative binomial
    distribution with parameters $(\alpha, p)$ with probability
    $\pi$ and from the constant equal to 0 with probability
    $1-\pi$. The distribtion is thus

    \begin{equation*}
    P(Y = y) = \pi\frac{\Gamma(\alpha+y)}{\Gamma(\alpha)y!}
    p^{\alpha}(1-p)^y + (1-\pi)\delta_0(y),
    \end{equation*}

    \noindent
    where $\delta_0(y) = 1$ if and only if $y=0$. Similarly,
    we set $\delta_{0^r}(y_1, \ldots, y_r) = 1$ if and only
    if $y_1=\ldots =y_r = 0$ and we define the zero-inflated negative
    multinomial (ZINM) distribution by the formula

    \begin{eqnarray}
      P(Y_1 = y_1, \ldots, Y_r = y_r) = \pi
       \frac{\Gamma(\alpha+y_1+\ldots+y_r)}
       {\Gamma(\alpha)y_1!\ldots y_r!}
\label{eq:zinm}
       &p_0^{\alpha}& p_1^{y_1} \ldots p_r^{y_r} + \\
       &\;& (1-\pi)\delta_{0^r}(y_1, \ldots, y_r). \nonumber
    \end{eqnarray}

\subsection{ZINM parameter estimation}
\label{sec:zinm_parameter_est}

    Mixture distributions can be fitted using the EM algorithm,
    which gives a tractable solution. However,
    the search algorithm can be trapped in local maxima, in which
    case it does not return the maximum likelihood estimate. For
    this reason, we use another method.

    Suppose, as in section \ref{sec:param_est_nm}, that an IID random
    sample of size $n$ drawn from a zero-inflated negative multinomial
    is available. In order to maximize (\ref{eq:zinm}), we introduce
    $z(k_1, \ldots, k_r)$, representing the number of observations
    such that $Y_1 = k_1, \ldots, Y_r = k_r$. For convenience, we
    also introduce $z_0 = z(0, \ldots, 0)$. The log-likelihood
    is then written as

    \begin{eqnarray*}
      \ell &=& z_0 \log(\pi p_0^{\alpha}+1-\pi) + \\
      &\;&\sum_{k_1, \ldots, k_r \neq z_0}z(k_1, \ldots,k_r)
      \Big( \log(\pi) + \log \Gamma(\alpha+k_1+\ldots+k_r) - \\
      &\;&\log \Gamma(\alpha) +  \alpha \log(p_0) +
      k_1\log(p_1) + \ldots + k_r\log(p_r) \Big).
    \end{eqnarray*}

    We first differentiate $\ell$ with respect to $\pi$ in order
    to obtain a substitution expression that will yield equations
    independent of $\pi$.

    \begin{eqnarray}
      \frac{\partial\ell}{\partial\pi} =
      z_0 \frac{p_0^{\alpha}-1}{\theta p_0^{\alpha}+1-\pi}
      + (n-z_0)\frac{1}{\pi} &=& 0
      \mbox{, which leads to} \nonumber \\
\label{eq:sub_pi}
\frac{z_0}{\pi p_0^{\alpha}+1-\pi} &=&
      \frac{n-z_0}{\pi(1-p_0^{\alpha})}.
    \end{eqnarray}

    We now differentiate $\ell$ with respect to $p_0, \ldots, p_r$,
    and as in section \ref{sec:param_est_nm}, we need to use Lagrange
    multipliers.

    \begin{eqnarray}
      \frac{\partial\ell}{\partial p_0} &=&
      z_0 \frac{\pi \alpha p_0^{\alpha-1}}
      {\pi p_0^{\alpha}+1-\pi} +
      \frac{\alpha(n-z_0)}{p_0} = \mu, \nonumber \\
      \frac{\partial\ell}{\partial p_i} &=&
      \frac{y_i^*}{p_i} = \mu \; (i \neq 0),
    \end{eqnarray}

    \noindent
    where we have defined the statistic
    $y_i^* = \sum_{k_1, \ldots, k_r \neq z_0} z(k_1, \ldots, k_r) k_i$.
    From these equations and $p_0+\ldots+p_r = 0$ we obtain

    \begin{equation}
\label{eq:sub_mu}
      \mu = \frac{y_1^*+\ldots+y_r^*}{1-p_0}.
    \end{equation}

    Observe \textit{en passant} that the term
    $y_1^* + \ldots + y_r^*$ is simply equal to the sum of all the
    observations.
    Substituting (\ref{eq:sub_pi}) and (\ref{eq:sub_mu}) in the
    expression of $\partial\ell/dp_0$, we obtain 

    \begin{equation}
      f(p_0, \alpha) = \frac{\alpha(n-z_0)}{p_0(1-p_0^{\alpha})} -
      \frac{y_1^*+\ldots+y_r^*}{1-p_0} = 0.
    \end{equation}

    We now differentiate $\ell$ with respect to $\alpha$ and
    substitute (\ref{eq:sub_pi}) to obtain an equation that
    depends on $\alpha$ and $p_0$.

    \begin{eqnarray*}
      g(p_0, \alpha) &=& \frac{\partial\ell}{\partial\alpha} =
      z_0 \frac{\pi p_0^{\alpha}\log(p_0)}
      {\pi p_0^{\alpha}+1-\pi} \\
      &+& \sum_{k_1, \ldots, k_r \neq z0}
      z(k_1, \ldots, k_r) \Big(\psi(\alpha+k_1 + \ldots +k_r)
      -\psi(\alpha) +\alpha\log(p_0)\Big) \\
        &=& \frac{(n-z_0)\log(p_0)}{1-p_0^{\alpha}}
        -(n-z_0) \psi(\alpha) \\ 
        &+& \sum_{k_1, \ldots, k_r \neq z_0} z(k_1, \ldots, k_r)
        \psi(\alpha+k_1 + \ldots +k_r) = 0.
    \end{eqnarray*}

    We now need to find $p_0$ and $\alpha$ such that
    $f(p_0, \alpha) = g(p_0, \alpha) = 0$. This is done with the
    Newton-Raphson method, which, in the case of several equations
    requires the Hessian matrix $H$ of the system. By definition

    \begin{equation*}
      H(p_0,\alpha) = \left(
      \begin{array}{ll}
        \partial f/\partial p_0 &
        \partial f/\partial \alpha  \\
        \partial g/\partial p_0 &
        \partial g/\partial \alpha  \\
      \end{array}
      \right).
    \end{equation*}

    The update formula is analogous to the case of a single equation,
    namely

    \begin{equation*}
      \left(
        \begin{array}{ll} p_0^{(t+1)} \\
        \alpha^{(t+1)} \end{array} \right) =
      \left(
      \begin{array}{ll} p_0^{(t)} \\ \alpha^{(t)} \end{array} \right) -
        H(p_0^{(t)}, \alpha^{(t)})^{-1} \left(
      \begin{array}{ll} f(p_0^{(t)},\alpha^{(t)}) \\
      g(p_0^{(t)},\alpha^{(t)})
      \end{array} \right).
    \end{equation*}

    The terms of the Hessian matrix are computed directly by
    differentiating $f$ and $g$ with respect to $p_0$ and
    $\alpha$.

    \begin{eqnarray*}
      \frac{\partial f(p_0,\alpha)}{\partial p_0} &=&
      -(n-z_0)\alpha \frac{1-(\alpha+1)p_0^{\alpha}}
      {(p_0(1-p_0^{\alpha}))^2} - \frac{y_1^*+\ldots+y_r^*}
        {(1-p_0)^2} \\
      \frac{\partial g(p_0,\alpha)}{\partial \alpha} &=&
      \frac{(n-z_0)(\log(p_0))^2p_0^{\alpha}}
      {(1-p_0^{\alpha})^2} -(n-z_0)\psi'(\alpha) \\
      &+& \sum_{k_1,\ldots,k_r \neq z_0}z(k_1, \ldots, k_r)
      \psi'(\alpha+k_1+\ldots+k_r) \\
      \frac{\partial f(p_0,\alpha)}{\partial \alpha} &=&
      \frac{\partial g(p_0,\alpha)}{\partial p_0} =
      (n-z_0)\frac{1-p_0^{\alpha}+\alpha p_0^{\alpha} \log(p_0)}
      {p_0(1-p_0^{\alpha})^2}.
    \end{eqnarray*}

    As a gradient method, the Newton-Raphson only guarantees
    convergence to a local maximum, the choice of the initial values
    is therefore important. To maximize the chances of finding the
    global maximum, the procedure described above is applied to
    a range of initial conditions. To determine these conditions,
    the number of all-null observations $z_0$ is decreased
    artificially, and the resulting dataset is fitted by the
    procedure descrbed in \ref{sec:param_est_nm} which gives a pair
    $(p_0, \alpha)$. The pairs corresponding to distinct values of
    $z_0$ are collected and used as initial conditions for the
    algorithm described in this section.

    In this method, there are only two free parameters, making
    it easier to find suitable intial conditions compared to
    the EM algorithm, in which there are three.

    
\section{An emission model for ChIP-seq}

    The readout of ChIP-seq and similar experiments is a sequence of
    reads mapped to genomic windows of identical size.
    The zero-inflated negative multinomial distribution is a good
    choice\footnote{One
    of the main weaknesses of that model is that it
    assumes that the distribution of the parameter $\lambda$ is IID
    for all genomic windows. This is probably not the case, as for
    every profile we expect that two neighboring windows have similar
    expected read counts.} to describe
    the number of reads per window for the following reasons:

    \begin{enumerate}
      \item It is a discrete random variables with values in
        $\mathbb{N}$.
      \item It is multidimensional and can thus accomodate multiple
        replicates.
      \item The marginal distributions are correlated.
      \item Unmappable regions of the genome inflate the windows
        with no read.
      \item Section \ref{sec:nb} shows that it can be interpreted
        as a Poisson distribution where the parameter $\lambda$
        varies as a Gamma variable. With this interpretation, each
        genomic window has a different expected read number.
        Conditionally on that number, the read count for a given
        window is a Poisson variable.
    \end{enumerate}

    We further assume that $r$ experiments are available.
    For a given genomic window and a given state $x_i$, the
    probability of observing $(y_1, \ldots, y_r)$ reads in the
    available profiles is

    \begin{equation*}
      \pi \frac{\Gamma(\alpha+y_1+\ldots+y_r)}
      {\Gamma(\alpha)y_1! \ldots y_r!}
      p_{0,i}^{\alpha} \; p_{1,i}^{y_1} \ldots p_{r,i}^{y_r}
      + (1-\pi) \delta_{0^r}(y_1, \ldots, y_r).
    \end{equation*}

\subsection{Estimating $\pi$ and $\alpha$}

    In section \ref{sec:gamma_poisson}, we have argued that the
    negative multinomial distribution can be seen as a Gamma-Poisson
    process, where $\alpha$ is the shape parameter of the underlying
    Gamma distribution. The interpretation in the context of ChIP-seq
    experiments is that genomic windows have different expected
    read counts because of experimental and computational artifacts.
    Similarly, the parameter $\pi$ is the probability that a genomic
    window is not mappable at all and will always have read count 0.
    These variations are a property of the genome and the methods
    used for the experiment, and not of a particular replicate.
    In terms of the formalism presented in section
    \ref{sec:HMM_formalism}, those values are independent of the
    of the HMM, and they can be estimated independently.

    The estimation of $\pi$ and $\alpha$ is based on the negative
    controls. These experiments provide the baseline variation of
    read count in the given genome with the given experimental setup.
    Section \ref{sec:zinm_parameter_est} provides a method to
    estimate $\pi$ and $\alpha$, together with the parameters
    $p_0, p_1, \ldots, p_c$, where $c$ is the number of available
    control experiments.

    If the values of $\pi$ and $\alpha$ can be considered constant,
    this is not the case of $p_0, p_1, \ldots, p_c$. Indeed, there
    are $r$ profiles to be modelled by a zero-inflated multinomial
    distribution, and the constrain $p_0 + p_1 + \ldots + p_r = 1$
    imposes that the values estimated on the negative controls
    alone cannot remain the same when all the profiles are
    considered. For this reason, we refer to estimates based on the
    controls only as $p_0^*, p_1^*, \ldots, p_c^*$.

    In section \ref{sec:marginal_nm}, we have shown that
    $p_s/p_0 = p_s^*/p_0^*$ $(1 \leq s \leq c)$, so even if the
    values have to be updated, their respective ratios have to be
    maintained. Note that those constrains also determine the value
    of the ratio $p_s / p_u$ for every pair $(s,u)$ where $s \leq c$
    and $u \leq c$. So instead of storing the values
    $p_0^*, p_1^*, \ldots, p_c^*$, at the end of the procedure
    presented in section \ref{sec:zinm_parameter_est}, we store the
    ratios $R_1=p_1^*/p_0^*, \ldots, R_c=p_c^*/p_0^*$.

\subsection{Estimating state-dependent parameters}

    The remaining parameters are state-dependent, which means that
    their value depends on the state of the HMM. For this reason,
    $p_{c+1}, \ldots, p_r$ have to be further indexed by the state
    and they are therefore referred to as
    $p_{c+1,i}, \ldots, p_{r,i}$, where $1 \leq i \leq m$.

    This is done through the Baum-Welch algorithm described in
    section \ref{sec:Baum-Welch}. The last term of equation
    (\ref{eq:QEM}) strongly depends on the emission model, which
    is why the solution had to be deferred until here. To complete
    the cycle of the Baum-Welch algorithm, we need to maximize
    the expression

    \begin{equation*}
      \sum_{k=0}^n E_{\theta^{(t)}} \Big\{
      \log g_{i_k}(y_k, \theta) \Big| y_0, \ldots, y_n \Big\},
    \end{equation*}

    \noindent
    where $g_{i_k}(y_k, \theta)$ is the probability of the emission
    if the state of the HMM at step $k$ is $i_k$, and $\theta$ is
    the set of parameters. In this expression, $y_k$ is actually
    an $r$-dimensional vector $(y_{k,1}, \ldots, y_{k,r})$, where
    $y_{k,s}$ is the value of the $s$-th variable (\textit{i.e.}
    the read count in window $k$ for experiment $s$).
    This can now be replaced by the formula
    of the zero-inflated negative multinomial model introduced above.
    More specifically, if $(y_1, \ldots, y_r) \neq 0^r$, the
    log-likelihood of a single emission if the HMM is in state $i$ is

    \begin{align*}
      g_i(y_1, \ldots, y_r|\theta) =
      \log(\pi) &+ \log\Gamma(\alpha+y_1+\ldots+y_r) -
      \log\Gamma(\alpha) + \\
      \alpha\log(p_{0,i}) &+ y_1 \log(p_{1,i}) + \ldots +
      y_r \log(p_{r,i}),
    \end{align*}

    \noindent
    and if $(y_1, \ldots, y_r) = 0^r$ it is

    \begin{align*}
      g_i(y_1, \ldots, y_r|\theta) = \log(\pi p_{0,i}^{\alpha}+1-\pi).
    \end{align*}

    In each state $i$, remember that the parameters
    $p_1, \ldots, p_c$ satisfy the constrain $p_{s,i}/p_{0,i}=R_s$,
    $1 \leq s \leq c$.  In the first expression above,
    the constant log factorial terms
    have been removed because they do not depend on the state,
    and they are therefore neutral for the
    estimation process. As explained above, $\pi$ and
    $\alpha$ are also state-independent and they can be considered
    constant. The first term above can thus be simplified to

    \begin{equation*}
      \alpha\log(p_{0,i}) + y_1\log(p_{1,i}) + \ldots
      + y_r\log(p_{r,i}).
    \end{equation*}

    If we denote  $Z_0$ the set of indices $k$ such that
    $y_{k,1} = \ldots = y_{k,r} = 0$, the third term of
    (\ref{eq:QEM}) can finally be computed as

    \begin{align}
      \sum_{i=1}^m\sum_{k \notin Z_0} &\phi_{k|n}(i)
      \Big(\alpha\log(p_{0,i}) + y_{k,1}\log(p_{1,i}) + 
      \ldots + y_{k,r}\log(p_{r,i}) \Big) \nonumber \\
       &+ \sum_{i=1}^m\sum_{k \in Z_0} \phi_{k|n}(i)
      \log(\pi p_{0,i}^{\alpha} + 1-\pi)
\label{eq:expl_zinm}
    \end{align}

    We differentiate (\ref{eq:expl_zinm}) with respect to the parameters
    $p_{0,i}, \ldots, p_{r,i}$, which are bound by the constrains
    $p_{0,i} + \ldots + p_{r,i} = 1$ and $p_{s,i}=R_i p_{0,i}$
    $(1 \leq s \leq c)$. Starting with $p_{0,i}$, we obtain

    \begin{align}
      \frac{\partial \ell}{\partial p_{0,i}} &=
      \frac{\alpha}{p_{0,i}} A
      + \frac{\pi\alpha p_{0,i}^{\alpha-1}} {\pi p_{0,i}^{\alpha} +
      1-\pi} B \nonumber \\
      &= \mu - R_1 \lambda_1 - \ldots -R_c \lambda_c,
\label{eq:p0}
    \end{align}

    \noindent
    where $A = \sum_{k \notin Z_0} \phi_{k|n}(i)$ and
    $B =  \sum_{k \in Z_0} \phi_{k|n}(i)$. 
    The differentiation of (\ref{eq:expl_zinm}) with respect to the
    other variables yields the following equalities

    \begin{align}
      \frac{\partial \ell}{\partial p_{s,i}} = 
      \frac{1}{p_{s,i}} \sum_{k \notin Z_0} \phi_{k|n}(i) y_{k,s}
      = \left\{
      \begin{array}{ll}
      \mu + \lambda_s, &\mbox{ if } 1 \leq s \leq c \\
      \mu &\mbox{ otherwise.}
      \end{array}
      \right.
\label{eq:p1}
    \end{align}

    For convenience, we define the statistics
    $y_{i,s}^* = \sum_{k \notin Z_0} \phi_{k|n}(i) y_{k,s}$
    and the constant $C = 1 + R_1 + \ldots + R_c$.
    Using (\ref{eq:p0}) and (\ref{eq:p1}), we obtain the following
    expression

    \begin{align}
      C \mu &= \frac{R_1y_{i,1}^*}{p_{1,i}} +
        \ldots + \frac{R_1y_{i,c}^*}{p_{c,i}} + \frac{\alpha}{p_{0,i}}A
      + \frac{\pi\alpha p_{0,i}^{\alpha-1}} {\pi p_{0,i}^{\alpha} +
      1-\pi} B \nonumber \\
\label{eq:cmu}
      &= \frac{y_{i,1}^* + \ldots + y_{i,c}^*+\alpha A}{p_{0,i}}
      + \frac{\pi\alpha p_{0,i}^{\alpha-1}} {\pi p_{0,i}^{\alpha} +
      1-\pi} B.
    \end{align}

    Starting from $p_{0,i} + p_{1,i} + \ldots + p_{r,i} = 1$ we
    obtain

    \begin{equation*}
      p_{0,i} + \frac{1}{C\mu} (y_{i,c+1}^* + \ldots
      + y_{i,r}^*) = \frac{1}{C},
    \end{equation*}

    \noindent
    where $C\mu$ is as shown in equation (\ref{eq:cmu}). In summary,
    maximizing (\ref{eq:expl_zinm}) boils down to solving the
    following equation

    \begin{equation*}
      f(p_{0,i}) = p_{0,i} + E \left(\frac{D + \alpha A}
      {p_{0,i}} + \frac{\pi\alpha p_{0,i}^{\alpha-1}}
      {\pi p_{0,i}^{\alpha} + 1-\pi} B\right)^{-1} - \frac{1}{C} = 0,
    \end{equation*}

    \noindent
    where $D = y_{i,1}^* + \ldots + y_{i,c}^*$ and
    $E = y_{i,c+1}^* + \ldots + y_{i,r}^*$.
    However tedious to derive,
    this expression is a function of $p_{0,i}$ only and it can
    thus be solved numberically by the Newton-Raphson
    method. We need to compute $f'$ and use the update formula
    $p_{0,i}^{(t+1)} = p_{0,i}^{(t)} -
    f(p_{0,i}^{(t)})/f'(p_{0,i}^{(t)})$.

    
    \begin{align*}
      f'(p_{0,i}) &= 1 - E
      \left( \frac{D +\alpha A}{p_{0,i}}
      + \frac{\pi\alpha p_{0,i}^{\alpha-1}}
      {\pi p_{0,i}^{\alpha} + 1-\pi} B\right)^{-2} \\
      &\times \left( \frac{(1-\pi)\pi\alpha(\alpha-1)p^{\alpha-2}
        -\pi^2\alpha p^{2\alpha-2}}{(\pi p_{0,i}^{\alpha}+1-\pi)^2}B
      -\frac{D + \alpha A}{p_{0,i}^2}
       \right).
    \end{align*}

    Once the value of $p_{0,i}$ has been computed, the
    $p_{s,i}$ can be computed via $p_{s,i} = R_i p_{0,i}$ for
    $1 \leq s \leq c$ and $p_{s,i} = y_{i,s}^* / \mu$ for
    $c+1 \leq s \leq r$,
    where $\mu$ is available from (\ref{eq:cmu}). The marks
    the end of the Baum-Welch algorithm and allows to perform
    another cycle.

\newpage
\begin{appendices}
\section{Estimation via the EM algorithm}

    Many mixture distributions can be fitted via the EM algorithm.
    The direct method implemented in Zerone is faster, but we
    nevertheless show how the EM algorithm can be used to fit
    the ZINB distribution, as this may be generally useful.

    By definition of the ZINB, numbers are drawn from a negative
    binomial distribution with parameters $(\alpha, p)$ with probability
    $\pi$ and from the constant equal to 0 with probability
    $1-\pi$. The distribtion is thus

    \begin{equation*}
\label{ZINB}
    P(Y = y) = \pi \frac{\Gamma(\alpha+y)}{\Gamma(\alpha)y!}
    p^{\alpha}(1-p)^y + (1-\pi)\delta_0(y).
    \end{equation*}

    We introduce a variable $Z$, which is equal to 1 if the numbers
    are drawn from the first distribution and 0 otherwise. The full
    likelihood distribution is

    \begin{equation*}
    P(y,z) = \pi\frac{\Gamma(\alpha+y)}{\Gamma(\alpha)y!}
    p^{\alpha}(1-p)^y\delta_1(z) +
    (1-\pi)\delta_0(y)\delta_0(z).
    \end{equation*}

    From this, we obtain
    \begin{equation*}
    P(Z=1|Y=y) =\frac{\pi\frac{\Gamma(\alpha+y)}{\Gamma(\alpha)y!}
    p^{\alpha}(1-p)^y}{
    \pi\frac{\Gamma(\alpha+y)}{\Gamma(\alpha)y!}
    p^{\alpha}(1-p)^y + (1-\pi)\delta_0(y)}.
    \end{equation*}

    In reality, we can substantially simplify this formula because
    it takes only two distinct values.

    \begin{equation}
\label{w}
    P(Z=1|Y=y) = \left\{
      \begin{array}{ll}
        w = \frac{\pi p^{\alpha}}{\pi p^{\alpha}+1-\pi}
        & \mbox{if } y = 0\\
        1 & \mbox{otherwise.}
      \end{array}
    \right.
    \end{equation}

    To compute the expected value of the full log-likelihood with
    respect to the conditional distribution of $Z$, we label the
    observations such that $y_1, \ldots, y_n$ are all positive, and
    that the remaining $n_0$ observations are 0.

    \begin{eqnarray*}
      \ell = n \log \Gamma(\pi) - n\log (\alpha) +
      n \alpha \log(p) +
      \sum_{k=1}^n \log \Gamma(\alpha + y_k) + y_k \log (1-p) \\
      + n_0 \left( w \log(\pi p^{\alpha}) +
    (1-w) \log(1-\pi) \right).
    \end{eqnarray*}

    Differentiating with respect to $\pi$ and $p$, we obtain

    \begin{eqnarray}
\label{update_pi}
      \pi &=& \frac{n+n_0w}{n+n_0} \\
\label{update_p}
      p &=& \frac{\alpha}{\alpha + y^*}, 
    \end{eqnarray}

    \noindent
    where $y^* = \sum_{k=1}^ny_k/(n+n_0w)$.
    Differentiating with respect to $\alpha$ and using the second
    equality above, we obtain the following equation, which can be
    solved numerically with the method of Newton-Raphson.

    \begin{equation*}
      f(\alpha) = -n \psi(\alpha) + (n+n_0w) \log \left(\frac{\alpha}{y^*
      + \alpha} \right) + \sum_{k=1}^n \psi(\alpha + y_k)
    \end{equation*}

    \begin{equation*}
      f'(\alpha) = -n \psi'(\alpha) + (n+n_0w)
      \frac{y^*}{\alpha(y^*+\alpha)} + \sum_{k=1}^n \psi'(\alpha + y_k)
    \end{equation*}

  Thus, assuming that $\alpha^{(t)}$, $\pi^{(t)}$ and $p^{(t)}$ are the
  values of the parameters at cycle $t$, we can compute their values
  at cycle $t+1$ with the following EM scheme.

  \begin{enumerate}
  \item Compute $w$ with formula (\ref{w}), where $\alpha$, $\pi$
    and $p$ are replaced by $\alpha^{(t)}$, $\pi^{(t)}$ and $p^{(t)}$,
    respectively. Compute $y^*$ using the value of $w$.
  \item Solve $f(\alpha) = 0$ using the Newton-Raphson method.
  \item Set $\alpha^{(t+1)}$ to the solution found in the previous
    step. Set $\pi^{(t+1)}$ and $p^{(t+1)}$ using equations
    (\ref{update_pi}) and (\ref{update_p}), respectively.
  \end{enumerate}

Those three steps can be iterated until convergence of the parameters,
where the final values are the EM estimates.

\end{appendices}

\end{document}
