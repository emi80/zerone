\documentclass[12pt]{article}
\usepackage{url}
\usepackage{hyperref}
\usepackage{mdframed}
\title{Zerone tutorial}
\author{Pol Cusc\'o and Guillaume Filion}

\begin{document}
\maketitle

\section{Building instructions}

Zerone is available as a Linux command line application and as an R package.

\subsection{Downloading}

We recommend that you use git to keep Zerone updated. You can clone the
repository from Github with the following command on a standard terminal.

\begin{verbatim}
git clone git@github.com:nanakiksc/zerone
\end{verbatim}

Note that this requires that you already have a Github account
and that the computer you are working on has an SSH key registered
on GitHub. If this is not the case, follow the instructions from
\url{https://help.github.com/articles/generating-ssh-keys/}.

Alternatively, if you prefer not to use git, you can download the source code
from \url{https://github.com/nanakiksc/zerone/archive/master.zip} with the
following commands.

\begin{verbatim}
wget https://github.com/nanakiksc/zerone/archive/master.zip
unzip zerone-master.zip
mv zerone-master zerone
\end{verbatim}

This should create a directory named \texttt{zerone}.

\subsection{Compiling}

To build Zerone, execute the following from the \texttt{zerone} directory.

\begin{verbatim}
cd zerone
make
\end{verbatim}

This should succeed on most Linux systems because \texttt{make} is
available by default. If this is not the case, you can obtain it by
typing \texttt{sudo apt-get install make} on the Ubuntu terminal.
Calling \texttt{make} should create an executable called \texttt{zerone}.



\subsection*{Installing the Zerone R package}

To install the Zerone R package, simply run this command from the
\texttt{zerone} directory.

\begin{verbatim}
R CMD INSTALL ZeroneRPackage
\end{verbatim}

Note that you need to have R installed on your computer. If this is not the
case, run the command \texttt{sudo apt-get install r-base} on Ubuntu.
If applicable, you may also append \texttt{sudo} before the command
to install the package system-wide.

\section{Zerone basics}

\subsection{Running Zerone}

To run Zerone, you have to specify the files that contain the mapped reads
of the ChIP-seq experiment you want to discretize. These can be in BED, SAM/BAM
and GEM (.map) formats and can be gzipped. You can include as many mock
control files and as many experimental replicates as you need, provided there
is at least one mock control and one experimental replicate. Just enter
mock controls after the \texttt{-0} or \texttt{--mock} option, and
targets after the \texttt{-1} or \texttt{--chip} option.

For example, you can type in the following commands from the \texttt{zerone}
directory.

\begin{verbatim}
./zerone --mock data/mock.sam --chip data/ctcf1.sam,data/ctcf2.sam
\end{verbatim}

Note that path expansion will not work when using comma separated file
names, so if you want to use path names starting with $\sim$, you can
simply specify the mock and chip options as many times as needed, as
shown below.

\begin{verbatim}
./zerone -0 data/mock.sam -1 data/ctcf1.sam -1 data/ctcf2.sam
\end{verbatim}

Enter the option \texttt{-h} or \texttt{--help} for usage instructions and
\texttt{--version} to print the version number.

\subsection{Window output}

Running any of the examples above should produce an output like
the following.

\begin{verbatim}
# QC score: 1.723
# features: 0.482, 10.000, 0.017, 0.407, 0.857
# advice: accept discretization.
chr1    1       300     0   0   0   0    0.30024
chr1    301     600     0   0   0   0    0.15942
chr1    601     900     0   0   0   0    0.09334
chr1    901     1200    0   0   0   0    0.06227
...
chr1    998701  999000  1   1   11  4    0.97852
chr1    999001  999300  1   2   72  34   1.00000
chr1    999301  999600  1   2   128 69   1.00000
chr1    999601  999900  1   0   10  5    0.99784
\end{verbatim}

\begin{mdframed}
The first three lines contain the result of the \textbf{quality control}.
It consists of a quality score followed by the numeric values of 5
features, and an advice to either \textbf{accept or reject} the
discretization. The recommendation is to accept if the score is positive
and reject if it is negative. If the score is higher than 1 (or lower
than -1), the advice is considered extremely reliable.
\end{mdframed}

The five numeric values of the features are not important per se.
They are used to compute the quality score. If only one replicate is
available, the last value and the QC score will be \texttt{-nan},
meaning that the quality control cannot be performed.

The rest of the lines contain the discretization proper.
The first three columns specify the chromosome, start and end
positions of each window.

\begin{mdframed}
The fourth column represents the \textbf{enrichment}. Zerone classifies
each window into one of two possible states. State 0 represents
background signal and state \textbf{1 represents an enriched window}.
\end{mdframed}

The fifth column contains the read counts of all the control profiles
summed together per window. The following columns show the number of
reads per window in the ChIP-seq files, in the order they were provided.
The final column is the estimated probability that the window is a
target (it is a confidence score for the call).

\subsection{List output}

With the \texttt{-l} or \texttt{--list-output} option, Zerone produces
an alternative output in which only the targets are shown after
merging consecutive windows. For instance, when running the following
command

\begin{verbatim}
./zerone -l -0 data/mock.sam -1 data/ctcf1.sam,data/ctcf2.sam
\end{verbatim}

\noindent
you should obtain the output shown below.

\begin{verbatim}
# QC score: 1.723
# features: 0.482, 10.000, 0.017, 0.407, 0.857
# advice: accept discretization.
chr1  237601  238200  1.00000
chr1  521401  522000  0.99969
chr1  567301  567900  0.91861
...
chr1  975901  976500  1.00000
chr1  990001  990600  0.99076
chr1  994201  995400  1.00000
chr1  998701  999900  1.00000
\end{verbatim}

The first three columns are the same as in window output \textit{i.e.}
chromosome or sequence name, start and end. The last column is the
confidence score of the called target. It is the \textit{highest}
confidence of the windows merged in the same target region.

\subsection*{The Zerone R package}

Once you have installed the Zerone R package, you can load it in
your R session with the command

\begin{verbatim}
library(zerone)
\end{verbatim}

The package contains a single function called \texttt{zerone()}.
You can display the help page with the command \texttt{?zerone}.
The object \texttt{ZeroneExampleData} is an example of
\texttt{data.frame} properly formatted for \texttt{zerone()}.
The command below runs \texttt{zerone()} on this data and
returns the Viterbi path (inferred sequence of states).

\begin{verbatim}
path <- zerone(ZeroneExampleData)
\end{verbatim}

If you want to return the parameters fitted by \texttt{zerone()},
you can do it as shown below.

\begin{verbatim}
listinfo <- zerone(ZeroneExampleData, returnall=TRUE)
\end{verbatim}

The list object \texttt{listinfo} then contains all the associated
parameters and extra information.

\section{Troubleshooting}

In case Zerone crashes, recompile it in debug mode. To do so,
run the following commands from the \texttt{zerone} directory.

\begin{verbatim}
make clean
make debug
\end{verbatim}

Then repeat the call that triggered the crash and contact
\href{mailto:guillaume.filion@gmail.com}{guillaume.filion@gmail.com}
attaching the debug information.

If you are using Linux, you can also run the unit tests to check
that the code behaves as expected on your machine. For this,
run the following commands in the Zerone directory.

\begin{verbatim}
make test -C src/test
\end{verbatim}

If everything works well, you should see the following output

\begin{verbatim}
********************************
*      Zerone unit tests       *
********************************
hmm/fwdb                      OK
hmm/fwdb (NAs)                OK
hmm/fwdb (underflow)          OK
hmm/block_fwdb                OK
hmm/block_fwdb (NAs)          OK
...
zerone/zinm_prob              OK
zerone/bw_zinm                OK
zerone/update_trans           OK
zerone/eval_bw_f              OK
********************************
\end{verbatim}

The tests will not run on every machine (in particular they will
not run on Mac), but this does not mean that something is broken.
If \texttt{Zerone unit tests} in the header does not appear, it
means that the tests cannot be compiled on your machine, but this
does not say anything about Zerone. If one of the tests fails,
\texttt{OK} will be replaced by \texttt{FAIL} and the test harness
will give some more information about what went wrong. You can contact
\href{mailto:guillaume.filion@gmail.com}{guillaume.filion@gmail.com}
attaching the test results so that we can help you troubleshoot.

You can also share your issues directly on the Github repository
at \url{https://github.com/nanakiksc/zerone/issues}. Be sure to give
enough information so that we can reproduce the issue.

Finally, we provide a Docker image with a running implementation
of Zerone. This can serve as an environment to run Zerone if it
is not available on your machine, or as an example to see how to
configure your own machine. The Docker image is available at
\url{https://hub.docker.com/r/nanakiksc/zerone/}.

\end{document}
