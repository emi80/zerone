\documentclass[12pt]{article}
\usepackage{url}
\usepackage{hyperref}
\usepackage{mdframed}
\title{Zerone tutorial}
\author{Pol Cusc\'o and Guillaume Filion}

\begin{document}
\maketitle

\section{Building instructions}

Zerone is available as a Linux command line application and as an R package.

\subsection{Downloading}

We recommend that you use git to keep Zerone updated. You can clone the
repository from Github with the following command on a standard terminal.

\begin{verbatim}
git clone git@github.com:gui11aume/zerone
\end{verbatim}

Note that this requires that you already have a Github account
and that the computer you are working on has an SSH key registered
on GitHub. If this is not the case, follow the instructions from
\url{https://help.github.com/articles/generating-ssh-keys/}.

Alternatively, if you prefer not to use git, you can download the source code
from \url{https://github.com/gui11aume/zerone/archive/master.zip} with the
following commands.

\begin{verbatim}
wget https://github.com/gui11aume/zerone/archive/master.zip
unzip zerone-master.zip
mv zerone-master zerone
\end{verbatim}

This should create a directory named \texttt{zerone}.

\subsection{Compiling}

To build Zerone, execute the following from the \texttt{zerone} directory.

\begin{verbatim}
cd zerone
make
\end{verbatim}

This should succeed on most Linux systems because \texttt{make} is
available by default. If this is not the case, you can obtain it by
typing \texttt{sudo apt-get install make} on the Ubuntu terminal.

Calling \texttt{make} should create an executable called \texttt{zerone}.

\subsection{Testing}

To check that the building was successful, test Zerone with
the following commands.

\begin{verbatim}
make test -C src/test
\end{verbatim}

If it passes the tests without any error message, then everything went fine
and you are done with the build. If not, something went
wrong. In this case, you can explain how to reproduce the problem
on \url{https://github.com/gui11aume/zerone/issues}.

\subsection*{Installing the Zerone R package}

To install the Zerone R package, simply run this command from the
\texttt{zerone} directory.

\begin{verbatim}
R CMD INSTALL ZeroneRPackage
\end{verbatim}

Note that you need to have R installed on your computer. If this is not the
case, run the command \texttt{sudo apt-get install r-base} on Ubuntu.

\section{Zerone basics}

\subsection{Running Zerone}

To run Zerone, you have to specify the files that contain the mapped reads
of the ChIP-seq experiment you want to discretize. These can be in BED, SAM/BAM
and GEM (.map) formats and can be gzipped. You can include as many negative
control files and as many experimental replicates as you need. Just enter
controls after the \texttt{-0} or \texttt{--mock} option, and
targets after the \texttt{-1} or \texttt{--chip} option.

For example, you can type in the following commands from the \texttt{zerone}
directory.

\begin{verbatim}
./zerone --mock data/mock.sam --chip data/ctcf1.sam,data/ctcf2.sam
\end{verbatim}

Where file1.bam and file2.bam are negative controls done without antibody,
and file2.bam and file3.bam are two experimental replicates.

Note that path expansion will not work when using comma separated file names,
so if you want to use path names starting with $\sim$, you can simply
specify the mock and chip options as many times as needed, as shown below.

\begin{verbatim}
./zerone -0 data/mock.sam -1 data/ctcf1.sam -1 data/ctcf2.sam
\end{verbatim}

Enter the option \texttt{-h} or \texttt{--help} for usage instructions and
\texttt{--version} to print the version number.

\subsection{Output}

Running Zerone as shown before should produce an output like the following.

\begin{verbatim}
# QC score: 1.289
# advice: accept discretization.
chr1    1       300     1   0   0   0
chr1    301     600     1   0   0   0
chr1    601     900     1   0   0   0
chr1    901     1200    1   0   0   0
...
chr1    998701  999000  2   1   11  4
chr1    999001  999300  2   2   72  34
chr1    999301  999600  2   2   128 69
chr1    999601  999900  2   0   10  5
\end{verbatim}

\begin{mdframed}
The first two lines contain the result of the \textbf{quality control}.
It consists of a quality score followed by an advice to either
\textbf{accept or reject} the discretization, if the score is positive or
negative, respectively. If the score is higher than 1 (or lower than -1), the
advice is extremely reliable.
\end{mdframed}

The rest of the lines contain the discretization proper. The first three columns
specify the chromosome, start and end positions of each window.

\begin{mdframed}
The fourth column represents the \textbf{enrichment}. Zerone classifies
each window into one of three possible states. States 0 and 1 represent two
types of background signal. State \textbf{2 represents an enriched window}.
\end{mdframed}

The fifth column contains the read count of all the control profiles summed
together, and the rest of the columns contain the read count of each of the
target profiles.

\subsection{List output}

With the \texttt{-l} or \texttt{--list-output} option, Zerone produces an
alternative output in which only enriched windows are shown. Also, all
contiguous windows are merged together.

\begin{verbatim}
# QC score: 1.289
# advice: accept discretization.
chr1    237601  238200
chr1    521401  522000
chr1    567301  567900
chr1    714001  714900
...
chr1    975901  976500
chr1    990001  990600
chr1    994201  995400
chr1    998701  999900
\end{verbatim}

\pagebreak

\subsection*{The Zerone R package}

You can load the package in R with this line.

\begin{verbatim}
library(zerone)
\end{verbatim}

Type \texttt{?zerone} from an R session to see the documentation.

\section{Troubleshooting}

In case Zerone crashes, start by recompiling it in debug mode. To do so,
run the following commands from the \texttt{zerone} directory.

\begin{verbatim}
make clean
make debug
\end{verbatim}

Then repeat the actions that triggered the crash and contact
\href{mailto:guillaume.filion@gmail.com}{guillaume.filion@gmail.com}
attaching the debug information.

\end{document}
